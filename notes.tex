\documentclass[10pt]{article}

\usepackage{amsmath}
\usepackage{amssymb}
\usepackage[margin = 2cm]{geometry}
\usepackage{graphicx}

\usepackage{sectsty}
\sectionfont{\fontsize{12}{16}\selectfont\centering}
\subsectionfont{\fontsize{10}{14}\selectfont}

\usepackage{natbib}
\bibliographystyle{apalike}

\usepackage{hyperref}
\hypersetup{colorlinks,linkcolor={blue},citecolor={blue},urlcolor={red}}  


\author{Gregor Steiner}
\title{Notes on Evans \& Didelez (2023)}

\begin{document}

\maketitle

This document collects my notes on \cite{evans_didelez_2023}. They propose a new parameterization of the distributions of interest, termed \textbf{frugal parameterization}, which consists of three pieces: the joint distribution of the treatment and covariates $p_{ZX}(z, x)$, the causal distribution of interest $p_{Y | X}^* (y|x)$, and a dependence measure between the outcome and the treatment conditional on the covariates $\phi_{YZ | X}^*$. In sequential treatment models \citep[see][Figure 2]{evans_didelez_2023}, this parameterization can avoid the so-called \textbf{g-null paradox} \citep{robins_wasserman_1997}.








\bibliography{references}

\end{document}
