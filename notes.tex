\documentclass[10pt]{article}

\usepackage{amsmath}
\usepackage{amssymb}
\usepackage[margin = 2cm]{geometry}
\usepackage{graphicx}

\usepackage{sectsty}
\sectionfont{\fontsize{12}{16}\selectfont\centering}
\subsectionfont{\fontsize{10}{14}\selectfont}

\usepackage{natbib}
\bibliographystyle{apalike}

\usepackage{hyperref}
\hypersetup{colorlinks,linkcolor={blue},citecolor={blue},urlcolor={blue}}  


\author{Gregor Steiner}
\title{Notes on Evans \& Didelez (2023)}

\begin{document}

\maketitle

This document collects my notes on \cite{evans_didelez_2023}. They propose a new parameterization of the distributions of interest, termed \textbf{frugal parameterization}, which consists of three pieces: the joint distribution of the treatment and covariates $p_{ZX}(z, x)$ (the 'past'), the causal distribution of interest $p_{Y | X}^* (y|x)$, and a dependence measure between the outcome and the covariates conditional on the treatment $\phi_{YZ | X}^*$. In sequential treatment models \citep[see][Figure 2]{evans_didelez_2023}, this parameterization circumvents the so-called \textbf{g-null paradox} \citep{robins_wasserman_1997}. The corresponding R-package \href{https://github.com/rje42/causl}{causl} provides functions to simulate from a frugal parametrization. \\

\textbf{Comments/Questions:}
\begin{itemize}
	\item Their example R2 is similar to problem 29.1 in \cite{ding2023course}.
\end{itemize}

\textbf{Further reading:} \cite{robins_wasserman_1997}, \cite{mcgrath2022revisiting}.








\bibliography{references}

\end{document}
