\documentclass[10pt]{article}

\usepackage{amsmath}
\usepackage{amssymb}
\usepackage[margin = 2cm]{geometry}
\usepackage{graphicx}

\usepackage{sectsty}
\sectionfont{\fontsize{12}{16}\selectfont\centering}
\subsectionfont{\fontsize{10}{14}\selectfont}

\usepackage{natbib}
\bibliographystyle{apalike}

\usepackage{hyperref}
\hypersetup{colorlinks,linkcolor={blue},citecolor={blue},urlcolor={blue}}  


\author{Gregor Steiner}
\title{Notes on Evans \& Didelez (2023)}

\begin{document}

\maketitle

This document collects my notes on \cite{evans_didelez_2023}. They propose a new parameterization for causal problems, termed \textbf{frugal parameterization}, which consists of three pieces: the joint distribution of the treatment and covariates $p_{ZX}(z, x)$ (the 'past'), the causal distribution of interest $p_{Y | X}^* (y|x)$, and a dependence measure between the outcome and the covariates conditional on the treatment $\phi_{YZ | X}^*$. In sequential treatment models, this parameterization circumvents the so-called \textbf{g-null paradox} \citep{robins_wasserman_1997}. Their main result shows that a frugal parametrization $\theta = (\theta_{ZX}, \theta_{Y|X}, \phi_{YZ | X})$ of the observational distribution induces a corresponding parameterization  $\theta^* = (\theta_{ZX}, \theta_{Y|X}^*, \phi_{YZ | X}^*)$ that is also frugal. Replacing $\theta_{ZX}$ in $\theta^*$ with $\eta_{ZX}(\theta_{ZX})$, where $\eta_{ZX}$ is a twice differentiable function with a Jacobian of constant rank, yields a parameterization of the causal joint distribution $p_{ZXY}^*$. Using this, they propose a rejection sampling algorithm to sample from $p_{ZXY}$ (implemented in the R-package \href{https://github.com/rje42/causl}{causl}). Furthermore, they show that under certain assumptions we can obtain consistent parameter estimates for the model $p_{Y | X}^*$ by maximizing the likelihood with respect to the observational data from $p_{ZXY}$. \\

\textbf{Comments/Questions:}
\begin{itemize}
	\item The parameterization is not unique: Given $p_{Y|X}^*$, $p_{ZX}$ and the dependence measure $ \phi_{YZ | X}^*$ need to be chosen. How do we choose them? Can this choice introduce uncertainty that is not accounted for, maybe like selective inference type problems?
	\item Similarly, what happens if the causal model $p_{Y|X}^*$ is misspecified? Then, the causal and observational joint distributions will be misspecified as well, leading to false simulation results, right?
\end{itemize}

\textbf{Further reading:} 
\begin{itemize}
	\item \cite{robins_wasserman_1997} and \cite{mcgrath2022revisiting} to better understand the g-null paradox. Also, problem 29.1 in \cite{ding2023course} is similar to their example R2.
\end{itemize}









\bibliography{references}

\end{document}
